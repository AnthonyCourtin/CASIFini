\section{A0. Introduction}
	Le sujet du projet est: Application Web Offline.\\
	Parmi les problématiques que l'on aura à traiter on trouvera : la tolérance aux pertes partielles ou totales de connexion, la synchronisation de référentiels de données, la gestion des conflits, le stockage des données, authentification, gestion des données.

\section{A1. Mots-clés}

	\begin{itemize}
		\item HTML5 ; 
		\item Application Cache ; 
		\item Offline Storage ;
		\item Mobile ;
		\item Indexed Database \& Web SQL Database ;
		\item Réplication asynchrone de données ; 
		\item Application Web Offline ;
		\item Manifest File ;
		\item Synchronisation ;
		\item Complexité.
	\end{itemize}

\section{A2. Webographie}

	\textbf{HTML5-CSS3}, \url{http://www.html5-css3.fr/html5/tutoriel-application-web-offline-html5-cache-manifest}\\

	\textbf{Blog Nouvelles Technologies}, \url{http://www.blog-nouvelles-technologies.fr/4116/comment-faire-pour-creer-une-application-web-hors-ligne-en-html5/}\\

	\textbf{W3 Schools}, \url{http://www.w3schools.com/html/html5_app_cache.asp}\\

	\textbf{Mozilla}, \url{https://developer.mozilla.org/fr/Apps/Build/Hors-ligne}\\

	\textbf{IBM}, \url{http://www.ibm.com/developerworks/xml/library/x-html5mobile3/index.html}

\section{A3. Bibliographie}

	Rodolphe Rimelé, \textit{HTML 5 - Une référence pour le développeur web}, Eyrolles, 7 mars 2013, 727 pages.
	\begin{itemize}
		\item Chapitre 15 - Stockage des données locales (Web Storage);
		\item Chapitre 16 - Bases de données (Indexed Database \& Web SQL Database);
		\item Chapitre 17 - Applications web hors ligne.
	\end{itemize}


