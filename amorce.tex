\section{Réplication asynchrone de données}

	La réplication est un processus de partage d'informations pour assurer la cohérence de données entre plusieurs sources de données redondantes, pour améliorer la fiabilité, la tolérance aux pannes, ou la disponibilité. On parle de réplication de données si les mêmes données sont dupliquées sur plusieurs périphériques.\\

	La réplication n'est pas à confondre avec une sauvegarde : les données sauvegardées ne changent pas dans le temps, reflétant un état fixe des données, tandis que les données répliquées évoluent sans cesse à mesure que les données sources changent. 

\section{Concepts HTML5}

	\subsection{Application Cache}
		Application cache met l’application en cache et permet à l’application d’être disponible sans connexion à internet. L’utilisateur peut donc utiliser l’application en hors ligne. Il permet d’accélérer la vitesse de l’utilisation de l’application car elle est en cache et de réduire le volume de données à télécharger auprès du serveur.

	\subsection{Offline Storage} 
		Offline Storage permet de sauvegarder les données et de faire fonctionner les applications ou les jeux sans être connectés à internet.

\section{API HTML5}
	
	\subsection{Web Storage}
		Les applications web peuvent être stockées en local dans le navigateur internet de l’utilisateur. Plus performant que les cookies.

	\subsection{Web SQL Database}
		Enregistrement des données dans une base de données pouvant être interrogée en SQL.

	\subsection{Indexed Database}
		La base de données contient des valeurs et des objets hiérarchisés. Un enregistrement se compose d’une clé et d’une valeur. Un index est implémenté pour optimiser l’accès aux données.

	\subsection{File Access}
		Représentation du fichier en objet dans l’application web. L’accès aux données se fait par les objets correspondants.

\section{Bases de données}
	Les principaux types de bases de données offrant des fonctionnalités de réplication dans un contexte Web sont les suivantes :

	\begin{itemize}
		\item MySQL ;;
		\item PostgreSQL
	\end{itemize}